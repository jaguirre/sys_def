%!TEX root =  ./sys_def.tex
\section{Polarization}

Polarization complicates things.  For antenna work, we generally work in relative polarization and we measure/compute co-polar and cross-polar beam patterns.  These are either orthogonal linear (which we are denoting E and N) or circular polarizations (RH/LH).
Although it is not how it is actually done, it is instructive to think of the antenna under test as fixed to looking at zenith and we run another transmitter antenna around on a great circle and measure what we get.

The Beam Jones matrix is the far-field vector E-field beam

Coordinate transforms of the beam Jones matrix require the rotation of a vector field on the sphere.  The resulting ``rotations'' of 

For imaging based deconvolution of polarized emission, the IXR is a useful figure of merit.

Putting the beam maximum response at spherical (0,0) (along the $z$-axis) does put a somewhat undesirable singularity in the description of the vector field there.  However, it does has the virtue of 

For the delay spectrum leakage, we can define a reasonable figure of merit as XXX